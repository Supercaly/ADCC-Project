\documentclass[submission,copyright,creativecommons]{eptcs}
\providecommand{\event}{ICE 2022} % Name of the event you are submitting to
\usepackage{breakurl}             % Not needed if you use pdflatex only.
\usepackage{underscore}           % Only needed if you use pdflatex.

\title{On implementing tuple spaces in Erlang}
%\author{Rob van Glabbeek
%\institute{NICTA\\ Sydney, Australia}
%\institute{School of Computer Science and Engineering\\
%University of New South Wales\thanks{A fine university.}\\
%Sydney, Australia}
%\email{rvg@cs.stanford.edu}
%\and
%Co Author \qquad\qquad Yet S. Else
%\institute{Stanford Univeristy\\
%California, USA}
%\email{\quad is@gmail.com \quad\qquad somebody@else.org}
%}
\def\titlerunning{On implementing tuple spaces in Erlang}
\def\authorrunning{}
\begin{document}
\maketitle

\begin{abstract}
Tuple space is a paradigm for parallel and distributed computing, made popular by its Linda implementation.
The idea of tuple space is very simple: it consists of a (distributed) repository where agents can read, write and withdrawn tuples in an atomic way. Tuples can contain arbitrary values, and are accessed by means of pattern-matching. Despite its simplicity, there is a lack of working implementations due to the fact that guaranteeng a distributed tuple space with atomic operations may lead to consistency and performances problems.  In this paper, we try to cope with this  problems by resorting to a programming language which is natively designed to work in distributed environments: Erlang. 
\end{abstract}

\section{Introduction}
% !TEX root = ice2022.tex

Several paradigms for parallel and distributed computing have been theorised and developed so far, and all of them differ either by the communication mechanism they use or by the level of abstraction they offer to the programmers. Among these, we can list those based on \emph{message passing} like remote procedure call (RPC), remote method invocation (RMI), distributed objects, message passing interface MPI; and those based on \emph{shared memory} like software transactional memory, tuple space~\cite{Gelernter85}, shared objects and  openMP~\cite{Mattson03}. Among these, the one which is less used is tuple space. Despite its simplicity, the lack of a reference implementation for this paradigm has prevented its wide spreading~\cite{BuravlevNM18}.


Tuple space is an abstraction of shared and distributed memory, made popular by the Linda programming language~\cite{Gelernter85}. Its idea is very simple and intuitive: agents communicate and synchonise each other by writing and reading data from a shared repository. This is also referred as the black-board metaphor. 
It consists of a (shared) repository of tuples (e.g., vectors of values) where agents can read, write or withdrawn (i.e., consume) tuples in an atomic way. 
The operations of reading and withdrawing use  \emph{pattern-matching} to access to the repository.
 These are also blocking operations:  if there are no tuples matching a particular pattern, then the agent performing the operation is blocked until a new matching tuple is produced (by another agent). The simplicity of this coordination model makes very intuitive and simple to implement various syncrhonisation mechanisms such as semaphores, synchronisation barriers, randezvous and so on (cf.,~\cite{Doberkat00b} Chapter 4). Let us consider the example depicted in Figure~\ref{fig:example}, the code of processes\footnote{Sometime we will refer to a process as an agent and viceversa.} $A$ and $B$ can be: 
 \definecolor{dkgreen}{rgb}{0,0.6,0}
\definecolor{gray}{rgb}{0.5,0.5,0.5}
\definecolor{mauve}{rgb}{0.58,0,0.82}
\setlength{\columnsep}{-1cm}
\lstset{ %
  language=erlang,                % the language of the code
  basicstyle=\footnotesize,           % the size of the fonts that are used for the code
  numbers=left,                   % where to put the line-numbers
  numberstyle=\tiny\color{gray},  % the style that is used for the line-numbers
  stepnumber=1,                   % the step between two line-numbers. If it's 1, each line 
                                  % will be numbered
  numbersep=5pt,                  % how far the line-numbers are from the code
  %backgroundcolor=\color{white},      % choose the background color. You must add \usepackage{color}
  showspaces=false,               % show spaces adding particular underscores
  showstringspaces=false,         % underline spaces within strings
  showtabs=false,                 % show tabs within strings adding particular underscores
  %frame=single,                   % adds a frame around the code
  rulecolor=\color{black},        % if not set, the frame-color may be changed on line-breaks within not-black text (e.g. commens (green here))
  tabsize=2,                      % sets default tabsize to 2 spaces
  captionpos=b,                   % sets the caption-position to bottom
  breaklines=true,                % sets automatic line breaking
  breakatwhitespace=true,        % sets if automatic breaks should only happen at whitespace
  title=\lstname,                   % show the filename of files included with \lstinputlisting;
                                  % also try caption instead of title
  keywordstyle=\color{blue},          % keyword style
  commentstyle=\color{dkgreen},       % comment style
  stringstyle=\color{mauve},         % string literal style
%   escapeinside={\%*}{*)},            % if you want to add LaTeX within your code
%   morekeywords={*,...}               % if you want to add more keywords to the set
}
\begin{multicols}{2}
\begin{lstlisting}[language=erlang]
process_A():
	out("sum",2,3)
\end{lstlisting}
\columnbreak
\begin{lstlisting}[firstnumber=3]
process_B(): 
	in("sum",x,y)
	out("res",x+y)	
\end{lstlisting}
\end{multicols}
 \noindent process $A$ produces, via the out operation, a tuple of the form $\tuple{\str{sum},2,3}$ indicating the fact it wants
 another process to perform the task $\str{sum}$. Process $B$ reads a tuple which satisfy the pattern $\tuple{\str{sum},x,y}$. This pattern will match any tuple with 3 elements whose
  whose first element is the exact string
 $\str{sum}$. Moreover, it  binds the second and third element of such a tuple to the variables $x$ and $y$, respectively. To compute the sum, process $B$ simply emits a tuple of the form $\tuple{\str{res},x+y}$. Let us not that what is shown in Figure~\ref{fig:example} is just one possible execution, since the pattern used by process $B$ at line
 $4$ could also match the tuple $\tuple{\str{sum},4,7}$.
 
 \begin{figure}
\centering
 	\includegraphics[scale=.2]{tuple_space}
	\label{fig:example}
	\caption{Example of tuple space}
 \end{figure}

\section{Ancillary files}

Authors may upload ancillary files to be linked alongside their paper.
These can for instance contain raw data for tables and plots in the
article, or program code.  Ancillary files are included with an EPTCS
submission by placing them in a directory \texttt{anc} next to the
main latex file. See also \url{https://arxiv.org/help/ancillary_files}.
Please add a file README in the directory \texttt{anc}, explaining the
nature of the ancillary files, as in
\url{http://eptcs.org/paper.cgi?226.21}.

\section{Prefaces}

Volume editors may create prefaces using this very template,
with {\tt $\backslash$title$\{$Preface$\}$} and {\tt $\backslash$author$\{\}$}.

\section{Bibliography}

We request that you use
\href{http://eptcs.web.cse.unsw.edu.au/eptcs.bst}
{\tt $\backslash$bibliographystyle$\{$eptcs$\}$}
\cite{bibliographystylewebpage}, or one of its variants
\href{http://eptcs.web.cse.unsw.edu.au/eptcsalpha.bst}{eptcsalpha},
\href{http://eptcs.web.cse.unsw.edu.au/eptcsini.bst}{eptcsini} or
\href{http://eptcs.web.cse.unsw.edu.au/eptcsalphaini.bst}{eptcsalphaini}
\cite{bibliographystylewebpage}. Compared to the original {\LaTeX}
{\tt $\backslash$biblio\-graphystyle$\{$plain$\}$},
it ignores the field {\tt month}, and uses the extra
bibtex fields {\tt eid}, {\tt doi}, {\tt eprint} and {\tt url}.
The first is for electronic identifiers (typically the number $n$
indicating the $n^{\rm th}$ paper in an issue) of papers in electronic
journals that do not use page numbers. The other three are to refer,
with life links, to electronic incarnations of the paper.

\paragraph{DOIs}

Almost all publishers use digital object identifiers (DOIs) as a
persistent way to locate electronic publications. Prefixing the DOI of
any paper with {\tt http://dx.doi.org/} yields a URI that resolves to the
current location (URL) of the response page\footnote{Nowadays, papers
  that are published electronically tend
  to have a \emph{response page} that lists the title, authors and
  abstract of the paper, and links to the actual manifestations of
  the paper (e.g.\ as {\tt dvi}- or {\tt pdf}-file). Sometimes
  publishers charge money to access the paper itself, but the response
  page is always freely available.}
of that paper. When the location of the response page changes (for
instance through a merge of publishers), the DOI of the paper remains
the same and (through an update by the publisher) the corresponding
URI will then resolve to the new location. For that reason a reference
ought to contain the DOI of a paper, with a life link to the corresponding
URI, rather than a direct reference or link to the current URL of
publisher's response page. This is the r\^ole of the bibtex field {\tt doi}.
{\bf EPTCS requires the inclusion of a DOI in each cited paper, when available.}

DOIs of papers can often be found through
\url{http://www.crossref.org/guestquery};\footnote{For papers that will appear
  in EPTCS and use \href{http://eptcs.web.cse.unsw.edu.au/eptcs.bst}
  {\tt $\backslash$bibliographystyle$\{$eptcs$\}$} there is no need to
  find DOIs on this website, as EPTCS will look them up for you
  automatically upon submission of a first version of your paper;
  these DOIs can then be incorporated in the final version, together
  with the remaining DOIs that need to found at DBLP or publisher's webpages.}
the second method {\it Search on article title}, only using the {\bf
surname} of the first-listed author, works best.  
Other places to find DOIs are DBLP and the response pages for cited
papers (maintained by their publishers).

\paragraph{The bibtex fields {\tt eprint} and {\tt url}}

Often an official publication is only available against payment, but
as a courtesy to readers that do not wish to pay, the authors also
make the paper available free of charge at a repository such as
\url{arXiv.org}. In such a case it is recommended to also refer and
link to the URL of the response page of the paper in such a
repository.  This can be done using the bibtex fields {\tt eprint}
or {\tt url}.  The latter field should \textbf{not} be used
to duplicate information that is also provided through {\tt doi} or {\tt eprint}.
You can find archival-quality URL's for most recently published papers
in DBLP, but please suppress repetition of DOI or {\tt eprint} information though {\tt url}.
In fact, it is often useful to check your references against DBLP records anyway,
or just find them there in the first place.

\paragraph{Typesetting DOIs and URLs}

When using {\LaTeX} rather than {\tt pdflatex} to typeset your paper, by
default no linebreaking within long URLs is allowed. This leads often
to very ugly output, that moreover is different from the output
generated when using {\tt pdflatex}. This problem is repaired when
invoking \href{http://eptcs.web.cse.unsw.edu.au/breakurl.sty}
{\tt $\backslash$usepackage$\{$breakurl$\}$}: it allows linebreaking
within links and yield the same output as obtained by default with
{\tt pdflatex}. 
When invoking {\tt pdflatex}, the package {\tt breakurl} is ignored.

The package {\tt $\backslash$usepackage$\{$underscore$\}$} is
recommended to deal with underscores in DOIs. This is not needed when
using {\tt $\backslash$usepackage$\{$breakurl$\}$} and not {\tt pdflatex}.

\paragraph{References to papers in the same EPTCS volume}

To refer to another paper in the same volume as your own contribution,
use a bibtex entry with
\begin{center}
  {\tt series    = $\{\backslash$thisvolume$\{5\}\}$},
\end{center}
where 5 is the submission number of the paper you want to cite.
You may need to contact the author, volume editors or EPTCS staff to
find that submission number; it becomes known (and unchangeable)
as soon as the cited paper is first uploaded at EPTCS\@.
Furthermore, omit the fields {\tt publisher} and {\tt volume}.
Then in your main paper you put something like:

\noindent
{\small \tt $\backslash$providecommand$\{\backslash$thisvolume$\}$[1]$\{$this
  volume of EPTCS, Open Publishing Association$\}$}

\noindent
This acts as a placeholder macro-expansion until EPTCS software adds
something like

\noindent
{\small \tt $\backslash$newcommand$\{\backslash$thisvolume$\}$[1]%
  $\{\{\backslash$eptcs$\}$ 157$\backslash$opa, pp 45--56, doi:\dots$\}$},

\noindent
where the relevant numbers are pulled out of the database at publication time.
Here the newcommand wins from the providecommand, and {\tt \small $\backslash$eptcs}
resp.\ {\tt \small $\backslash$opa} expand to

\noindent
{\small \tt $\backslash$sl Electronic Proceedings in Theoretical Computer Science} \hfill and\\
{\small \tt , Open Publishing Association} \hfill .

\noindent
Hence putting {\small \tt $\backslash$def$\backslash$opa$\{\}$} in
your paper suppresses the addition of a publisher upon expansion of the citation by EPTCS\@.
An optional argument like
\begin{center}
  {\tt series    = $\{\backslash$thisvolume[EPTCS]$\{5\}\}$},
\end{center}
overwrites the value of {\tt \small $\backslash$eptcs}.

\nocite{*}
\bibliographystyle{eptcs}
\bibliography{generic}
\end{document}
